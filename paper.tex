\documentclass[a4paper,11pt]{article}
\usepackage{sobrapo-template}
\usepackage[brazil]{babel}
\usepackage[latin1]{inputenc}
\usepackage{amsmath,amssymb}

\usepackage{tikz}
\usepackage{parskip}

\title{An Incomplete Overview of some Applications of Game Theory to Patient Flow}

\begin{document}

\maketitle

\author{
\name{Vincent Knight}
\institute{Cardiff University}
\iaddress{Cardiff, UK}
\email{knightva@cf.ac.uk}
}

\author{
\name{Paul Harper}
\institute{Cardiff University}
\iaddress{Cardiff, UK}
\email{harper@cf.ac.uk}
}

\author{
\name{Jeff Griffiths}
\institute{Cardiff University}
\iaddress{Cardiff, UK}
\email{griffiths@cf.ac.uk}
}

\author{
\name{Izabela Komenda}
\institute{Cardiff University}
\iaddress{Cardiff, UK}
\email{komendai@cf.ac.uk}
}

\author{
\name{Rob Shone}
\institute{Cardiff University}
\iaddress{Cardiff, UK}
\email{shoner@cf.ac.uk}
}

\vspace{8mm}

\begin{abstract}
Describe paper.
\end{abstract}

\bigskip
\begin{keywords}
Patient Flow, Game Theory, Queueing Theory

\bigskip
\noindent{Main Area: Healthcare}
\end{keywords}


\newpage

\section{Introduction}

\begin{itemize}
    \item Review GT in HC in general (small);
    \item Review congestion type games;
    \item PoA;
    \item Structure.
\end{itemize}

\section{Choosing queues}

This section describes how patient choices between various congestion affected services may be modelled.
In particular the situation shown diagrammatically in Figure \ref{fig:choices} is considered: patients have a choice amongst $M/M/c$ queues.

\begin{figure}[!hbtp]
Show diagram.
\end{figure}

There are two approaches to solving this problem: assuming that patients observe or not the system states before choosing a facility.
A rigorous comparison of these two approaches for individuals choosing to join an $M/M/1$ queue is given in \cite{}.

An unobservable study is given in \cite{} where routing games \cite{Rouhgharden} are used to study the system described.
The routing game used is shown in \ref{fig:routinggame}.

\begin{figure}[!hbtp]
Show routing game image.
\end{figure}

It can be shown that the cost for any given flow $\lambda$ (denoting the amount of traffic from source $i$ to facility $j$) corresponds to...

\begin{equation}
C(\lambda)=\sum_{i=1}^m\alpha_i\sum_{j}^nd_{ij}\lambda_{ij}+\sum_{j=1}^n\sum_{i=1}^m\lambda_{ij}w_j\left(\sum_{i=1}^m\lambda_{ij}\right)+\sum_{i=1}^m\beta_i\left(\Lambda_i-\sum_{j=1}^n\lambda_{ij}\right)
\label{C}\end{equation}

Obtaining the Nash flow: ie the equilibrium behaviour.

\begin{equation}
\Phi(\lambda)=\sum_{i=1}^m\alpha_i\sum_{j}^nd_{ij}\lambda_{ij}+\sum_{j=1}^n\int_0^{\sum_{i=1}^m\lambda_{ij}}w_j(x)dx+\sum_{i=1}^m\beta_i\left(\Lambda_i-\sum_{j=1}^n\lambda_{ij}\right)
\label{Phi}\end{equation}

To be able to obtain the PoA for a given instance the following mathematical program must be solved:

$$\begin{array}{l@{\hspace{2cm}}l}\text{OPTMP:}&\text{NASHMP:}\\
\text{minimise }(\ref{C})&\text{minimise }(\ref{Phi})
\end{array}$$
such that:
\begin{equation}
\sum_{j=1}^n\lambda_{ij}\leq\Lambda_{i}\text{ for all }i\in[m]\label{constraint 1}
\end{equation}
\begin{equation}
\lambda_{ij}\in\mathbb{R}^{m\times n}_{\geq 0}\text{ for all }i\in[m],\;j\in[n]\label{constraint 2}
\end{equation}
\begin{equation}
\sum_{i=1}^m\lambda_{ij}<c_j\mu_j\text{ for all }j\in[n]\label{constraint 3}
\end{equation}


The constant $\alpha_i\in\mathbb{R}_{\geq0}$ is simply a weighting statistic for the relative importance of travel distances to the other factors (once again allowing for this coefficient to be dependent on population partitioning).

In \cite{} various theoretical results are proven. With regards to the effect of worth of service on the PoA but also with regards to demand. The profile of Figure \ref{fig:poaprofile} is shown to hold in general.

\begin{figure}[!hbtp]
Pic
\end{figure}

To consider systems where individuals are able to observe the system there are two approaches: the first is to use a simulation based approach that allows individuals to choose their most desirable queue.
One such approach that was considered specifically in the context of healthcare was considered in \cite{}.

Given that individuals will consider a simple selfish decision rule this approach is relatively straightforward and can also be considered using straightforward analytical Markov models.
The difficulty with this approach is appears when attempting to obtain the PoA.
To carry this out an optimal policy must be obtained.

In \cite{} various dynamic programming and approximate dynamic programming techniques are proposed that are able to not only give an optimal policy but also prove the following observation:

\begin{center}
\textit{Selfish users make busier systems.}
\end{center}

In the next section selfish congestion related decisions by managers will be considered.

\section{CCU Work}

Talk about idea.
Talk about Markov model.
Talk about proof that pure strategy NE exists.
Show how can be used to influence t.

\section{EU and EMV Interface}

Describe simple Markov model for hospital.
Describe routing game model for thing.

\section{Conclusions}

\bigskip
\noindent{\bf Refer\^encias}

\noindent \textbf {Anna, A.} (1996), Comunica\c c\~ao, \textit{Atas do XXVIII SBPO}, 123-134.

\noindent \textbf{Gates, B.}, \textit{Um Livro Muito Bom}, Editor, Local, 2003.

\noindent  \textbf{Pel\'e, E. N. e Rom\'ario, R. R.}, Exemplo de artigo em livro, em Windows, P. T. e Linux, V. G. (Eds.),
\textit{Colet\^ania de Artigos},  Editor, Local, 123-133, 2004.

\noindent  \textbf{Silva, A. B., Souza, C. D. e Santos, E. F.}, T\'\i tulo de um relat\'orio t\'ecnico dispon\'\i vel na Internet,
\textit{ Relat\'orios de Pesquisa em Engenharia de Produ\c c\~ao}, n. 4,  Universidade Z,
1999

\noindent{(www.universidade.br/rel, 4, 2001.}

\noindent  \textbf{Smith, S. e Jones, J.} (2002), A paper on Operations Research, \textit{Pesquisa Operacional},
32, 5-44.

\end{document}
